\subsection{Procedure}
We utilized the 8-point algorithm as descried in the handout to estimate the fundamental matrix $F$. More specifically, given corresponding points $\textbf{x}_{1}$ and $\textbf{x}_{2}$, the fundamental matrix is defined as:

$$
\textbf{x}_{2}^{t} F \textbf{x}_{1} = 0.
$$

\subsubsection{Normalization} 
We first start by normalizing our corresponding points to minimize the effects of noise. We can achieve this by translating by $\mu$ (so points are shifted to have a mean at the origin) and scaling by $\sigma$ (so the mean distance to the origin is $\sqrt{2}$). This is achieved by transforming the points by a linear system $T$, where T is defined as the following:

$$
T =
\begin{bmatrix}
\sigma & 0 & -\sigma \mu \\
0 & \sigma & -\sigma \mu \\
0 & 0 & 1
\end{bmatrix}
$$

\subsubsection{Optimization}

\noindent As seen in the handout, the above formulation for $F$ (or $f$ as a row-wise reshaped column vector) can be approximated by optimizing the following:

\begin{equation}
\min_{f} \left\|Af\right\|_{2} \ \ \ s.t. \ \ \  \left\|f\right\|_{2} = 1
\end{equation}

\noindent We can now find $f^{*}$ which solves the above optimization problem by using the singular value decomposition of matrix $A$. Using SVD, $A = \tilde{U}\tilde{S}\tilde{V}^{T}$, where $\tilde{S}$ is a diagonal matrix with descending singular values along its diagonal, and $\tilde{U}$ and $\tilde{V}^{T}$ are orthogonal matrices. The columns of $\tilde{V}^{T}$ (or equivalently rows of $\tilde{V}$) correspond to the singular values of $S$. Therefore, the $f^{*}$ which solves equation (1) is the right singular vector (column in $\tilde{V}^{T}$) that corresponds to the smallest singular value in $\tilde{S}$. Then the estimate of the fundamental matrix, $F^{*}$ is achieved by reshaping the column vector $f^{*}$ row-wise into a $3x3$ matrix. \\

\noindent Furthermore, as seen in the handout, we can constrain our estimate of $F$ to be rank two by first taking the SVD of $F^{*}$: $F^{*}$ = $U S V^{T}$. Then, a rank of 2 is achieved by recomposing with 

$$
\hat{S} = diag(s1, s2, 0)
$$

\noindent where we set the least singular value of $F^{*}$ to zero. Then we achieve a approximate F of rank 2 by recomposing:

$$
F = U\hat{S}V^{T}
$$

\subsubsection{Denormalization}
Finally we need to denormalize our result. This is achieved by the following:

$$
F \leftarrow T_{2}^{t} F T_{1} 
$$

\subsection{Results}
The following are the fundamental matrices for the corresponding points of the 'house' and the 'library':
$$
F_{house} = 
\begin{bmatrix}
0 & 0 & 0 \\
0 & 0 & 0 \\
0 & 0 & 0 \\ 
\end{bmatrix}, \ \ \ \ 
F_{library} = 
\begin{bmatrix}
0 & 0 & 0 \\
0 & 0 & 0 \\
0 & 0 & 0 \\ 
\end{bmatrix}
$$

\subsection{Residuals}
The following table gives the residuals (mean squared distance between
the points in the two images and the corresponding epipolar lines) for 'house' and 'library':

\begin{table}[H]
\begin{center}
\begin{tabular}{|l|r|}
\hline
name & residual error\\
\hline
house & tbd \\
\hline 
library & tbd \\
\hline
\end{tabular}
\caption{residual error}
\label{Table:res_err}
\end{center}
\end{table}

\subsection{Additional Questions}